\documentclass[a4paper,12pt]{article}
\usepackage[paper=a4paper, hmargin=1.5cm, bottom=2.0cm, top=2.0cm]{geometry}
\usepackage[latin1]{inputenc}
\usepackage{graphicx}
\usepackage[T1]{fontenc}
\usepackage[spanish,activeacute]{babel}
\begin{document}
	\begin{center}
	\begin{Huge} Universidad de Buenos Aires \\\end{Huge}
	\begin{LARGE}Facultad de Ciencias Exactas \\ \end{LARGE}
	\begin{Large}Departamento de Computaci\'on \\ \end{Large}
	\vspace*{1cm}
	\begin{figure}[ht!]
		\begin{center}
			\includegraphics[width=60mm]{UBA.jpg}
		\end{center}	
	\end{figure}
	\begin{Large}Metodos Num\'ericos \\ \end{Large}
	\begin{Large}Trabajo Pr\'actico N°1 \end{Large}
	
	\vspace*{1cm}	
	
	\begin{tabular}{|c|c|c|}
		\hline
		Apellido y Nombre & LU & E-mail\\
		\hline
		Mar\'ia Candela Capra Coarasa & 234/11 & canduh\_27@hotmail.com\\
		Leandro Lovisolo            & 645/11 & leandro@leandro.me\\
		Lautaro Jos\'e Petaccio       & 443/11 & lausuper@gmail.com\\
		\hline
	\end{tabular}
	\end{center}
	\begin{large} \em Resumen: \end{large} \\
En este trabajo se implementaron los m\'etodos de Newton y bisecci\'on con el objetivo de realizar una aproximaci\'on de los par\'ametros de la distribuci\'on Gamma Generalizada estimando el par\'ametro $\beta$ a trav\'ez de una serie de muestras y luego utilizandolo para conseguir los par\'ametros $\lambda$ y $\sigma$ mediante el m\'etodo de los estimadores de m\'axima verosimilitud. \\
	\begin{large} \em Palabras claves: \end{large} bisecci\'on, Newton, gamma, estimar
\end{document}