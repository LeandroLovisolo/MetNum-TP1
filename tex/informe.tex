\documentclass[a4paper,10pt,twoside]{article}
\usepackage{a4wide}
\usepackage[utf8]{inputenc}
\usepackage[spanish,es-noshorthands,es-ucroman]{babel} % Opciones luego de spanish por problema de compatibilidad con csvsimple
\usepackage{setspace}        % Para setear el espacio entre lineas del título en la carátula
\usepackage{fancyhdr}
\usepackage{lastpage}
\usepackage{amsmath}
\usepackage{amsfonts}
\usepackage{verbatim}
\usepackage{graphicx}
\usepackage{blindtext}
\usepackage{floatrow}
\usepackage{csvsimple}
\usepackage{xstring}
\usepackage{xcolor}
\usepackage{listings}


%%%%%%%%%% Configuración de Listings - Inicio %%%%%%%%%%
\definecolor{mygray}{rgb}{0.5,0.5,0.5}
\lstset{extendedchars=\true}
\lstdefinestyle{customc}{
	belowcaptionskip=1\baselineskip,
	breaklines=true,
	language=C++,
	showstringspaces=false,
	basicstyle=\footnotesize\ttfamily,
	keywordstyle=\bfseries\color{green!40!black},
	commentstyle=\itshape\color{purple!40!black},
	identifierstyle=\color{blue},
	stringstyle=\color{orange},
	numbers=left,
	numberstyle=\color{mygray}
}
\lstset{escapechar=@,style=customc}
%%%%%%%%%% Configuración de Listings - Fin %%%%%%%%%%


%%%%%%%%%% Configuración de Fancyhdr - Inicio %%%%%%%%%%
\pagestyle{fancy}
\thispagestyle{fancy}
\lhead{Estimación de Parámetros de la Distribución $\Gamma$ Generalizada}
\rhead{Capra, Lovisolo, Petaccio}
\renewcommand{\footrulewidth}{0.4pt}
\cfoot{\thepage /\pageref{LastPage}}

\fancypagestyle{caratula} {
   \fancyhf{}
   \cfoot{\thepage /\pageref{LastPage}}
   \renewcommand{\headrulewidth}{0pt}
   \renewcommand{\footrulewidth}{0pt}
}
%%%%%%%%%% Configuración de Fancyhdr - Fin %%%%%%%%%%


%%%%%%%%%% Macros para incluir gráficos - Inicio %%%%%%%%%%
% Inserta un gráfico centrado.
% Uso: \grafico{ruta al grafico .tex}{etiqueta}
\newcommand{\grafico}[2]{
	\begin{figure}[H]
		\caption{#2}
		\centering
		\input{#1}
	\end{figure}
}

% Inserta dos gráficos en fila.
% Uso: \graficodoble{ruta al grafico 1.tex}{etiqueta 1}{ruta al grafico 2.tex}{etiqueta 2}
\newcommand{\graficodoble}[4]{
	\begin{figure}[H]
		\begin{floatrow}
			\floatbox{figure}[.4\textwidth][\FBheight][t]
			{\caption{#2}}
			{\input{#1}}
			\hspace*{1cm}
			\floatbox{figure}[.4\textwidth][\FBheight][t]
			{\caption{#4}}
			{\input{#3}}
		\end{floatrow}
	\end{figure}	
}
%%%%%%%%%% Macros para incluir gráficos - Fin %%%%%%%%%%


%%%%%%%%%% Macro para incluir CSVs - Inicio %%%%%%%%%%
% Inserta una tabla con el contenido de un archivo CSV.
% Uso: \incluircsv{ruta al archivo.csv}{formato de columnas}{cabeza}{filas}
%
% Parámetros:
%   {formato de columnas} formato idéntico al usado en \begin{table}{...}.
%   {cabeza} el contenido de la primera fila de la tabla.
%   {filas} el contenido de cada fila individual.
%
% Ejemplo:
%
%   --- Archivo ejemplo.csv ---
%
%     Nombre,Apellido,Edad
%     Juan,Perez,30
%     María,Gomez,25
%     Pablo,Rodríguez,38
%
%   --- Código LaTeX ---
%
%     \incluircsv{ejemplo.csv}
%                {|r|l|}
%                {Sujeto & Edad}
%                {\Nombre \Apellido & \Edad}
%
%   --- Salida ---
%
%      +-----------------+------+
%      |          Sujeto | Edad |
%      +-----------------+------+
%      |      Juan Perez | 30   |
%      |     María Gomez | 25   |
%      | Pablo Rodríguez | 38   |
%      +-----------------+------+
\newcommand{\incluircsv}[4]{
	\begin{center}
		\csvreader[tabular=#2,
		           table head=\hline #3 \\\hline,
		           late after line=\\,
		           late after last line=\\\hline,
		           head to column names]
		          {#1}{}{#4}
	\end{center}              
}
%%%%%%%%%% Macro para incluir CSVs - Fin %%%%%%%%%%


%%%%%%%%%% Macros varios - Inicio %%%%%%%%%%
% Devuelve el nombre de una muestra dada la ruta al archivo correspondiente
% Uso: \nombremuestra{data/Xn.txt}
%
% Ejemplo: \nombremuesra{data/X1.txt}
% Salida: X1
\newcommand{\nombremuestra}[1]{%
	\let \nombre #1%
	\StrSubstitute{\nombre}{data/}{}[\nombre]%
	\StrSubstitute{\nombre}{data/tag}{}[\nombre]%
	\StrSubstitute{\nombre}{.txt}{}[\nombre]%
	\nombre%
}
%%%%%%%%%% Macros varios - Fin %%%%%%%%%%


\begin{document}


%%%%%%%%%%%%%%%%%%%%%%%%%%%%%%%%%%%%%%%%%%%%%%%%%%%%%%%%%%%%%%%%%%%%%%%%%%%%%%%
%% Carátula                                                                  %%
%%%%%%%%%%%%%%%%%%%%%%%%%%%%%%%%%%%%%%%%%%%%%%%%%%%%%%%%%%%%%%%%%%%%%%%%%%%%%%%


\thispagestyle{caratula}

\begin{center}

Departamento de Computación,\\
Facultad de Ciencias Exactas y Naturales,\\
Universidad de Buenos Aires

\begin{spacing}{2.5}
\begin{Huge}
Estimación de Parámetros de la\\
Distribución Gamma Generalizada
\end{Huge}
\end{spacing}

\vspace{1cm}

Trabajo Práctico 1, \\
Métodos Numéricos, \\
Primer Cuatrimestre de 2013

\vspace{1cm}

\includegraphics[width=5cm]{UBA.jpg} 

\vspace{1cm}

\begin{tabular}{|c|c|c|}
\hline
Apellido y Nombre & LU & E-mail\\
\hline
María Candela Capra Coarasa & 234/11 & canduh\_27@hotmail.com\\
Leandro Lovisolo            & 645/11 & leandro@leandro.me\\
Lautaro José Petaccio       & 443/11 & lausuper@gmail.com\\
\hline
\end{tabular}

\end{center}

\textbf{Resumen:} \\
Introducimos un método para estimar los parámetros de una D$\Gamma$G a partir de una muestra.
Utilizamos estimadores de máxima verosimilitud y cálculo de momentos para obtener un sistema
de ecuaciones no-lineal de los parámetros de la distribución, y aplicamos métodos de análisis
numérico (bisección y Newton) para aproximar una solución con un margen de error dado.

\textbf{Palabras claves:} D$\Gamma$G, estimadores, bisección, Newton.

\vspace{6cm}

\newpage


%%%%%%%%%%%%%%%%%%%%%%%%%%%%%%%%%%%%%%%%%%%%%%%%%%%%%%%%%%%%%%%%%%%%%%%%%%%%%%%
%% Índice                                                                    %%
%%%%%%%%%%%%%%%%%%%%%%%%%%%%%%%%%%%%%%%%%%%%%%%%%%%%%%%%%%%%%%%%%%%%%%%%%%%%%%%


\tableofcontents

\newpage


%%%%%%%%%%%%%%%%%%%%%%%%%%%%%%%%%%%%%%%%%%%%%%%%%%%%%%%%%%%%%%%%%%%%%%%%%%%%%%%
%% Introducción Teórica                                                      %%
%%%%%%%%%%%%%%%%%%%%%%%%%%%%%%%%%%%%%%%%%%%%%%%%%%%%%%%%%%%%%%%%%%%%%%%%%%%%%%%


\section{Introducción Teórica}

\blindtext


%%%%%%%%%%%%%%%%%%%%%%%%%%%%%%%%%%%%%%%%%%%%%%%%%%%%%%%%%%%%%%%%%%%%%%%%%%%%%%%
%% Desarrollo                                                                %%
%%%%%%%%%%%%%%%%%%%%%%%%%%%%%%%%%%%%%%%%%%%%%%%%%%%%%%%%%%%%%%%%%%%%%%%%%%%%%%%


\section{Desarrollo}

Para obtener una primera noción de la ubicación del parámetro $\beta$ en la
recta real, aplicamos el método de bisección sobre todas las muestras en el
intervalo $[1, 100]$\footnote{Tomamos un intervalo arbitrariamente grande,
usando como heurística los parámetros de las muestras de referencia provistas
por la cátedra y la similitud del rango de valores de las mediciones de las
muestras de referencia y de las que se desean estimar.} con límites de 100
iteraciones y error absoluto $10^{-4}$, y precisión de 51 bits de mantisa.

En cada caso el método convergió a una solución luego de 20 iteraciones.
Una vez hallado $\beta$, despejamos los parámetros $\sigma$ y $\lambda$ y
dibujamos el histograma y función de densidad superpuestos. Se observa
excelente ajuste en todas las muestras.

Siguiendo un razonamiento similar, modificamos el experimento anterior,
esta vez aplicando el método de Newton, tomando como aproximación inicial
$\beta_0 = 10$ y manteniendo los mismos criterios de parada y precisión.

Observamos que el método converge a soluciones muy similares a las del
experimento inicial luego de 7 iteraciones en promedio, a excepción de la
muestra X3. Con esta última, el método converge a un valor muy pequeño de
$\beta$ que efectivamente anula la ecuación (4) pero no ajusta al histograma
de la muestra.

Para remendar esta situación, repetimos el experimento sobre la muestra X3
tomando como aproximación inicial la parte entera del valor de $\beta$ hallado
con el método de bisección. Observamos convergencia a una solución correcta
luego de 5 iteraciones.

Una vez hallados los parámetros de las distribuciones de cada muestra y un
conjunto de puntos de partida para los métodos que decidimos aplicar,
procedimos a explorar otras configuraciones haciendo variar el máximo error
tolerado y la cantidad de dígitos de precisión en la mantisa, intentando
reducir la cantidad de tiempo e interaciones necesarias para converger a una
solución y encontrar un equilibrio entre buen ajuste y tiempo de ejecución.


%%%%%%%%%%%%%%%%%%%%%%%%%%%%%%%%%%%%%%%%%%%%%%%%%%%%%%%%%%%%%%%%%%%%%%%%%%%%%%%
%% Resultados                                                                %%
%%%%%%%%%%%%%%%%%%%%%%%%%%%%%%%%%%%%%%%%%%%%%%%%%%%%%%%%%%%%%%%%%%%%%%%%%%%%%%%


\section{Resultados}

Incluimos a continuación los parámetros hallados para cada muestra junto
con sus respectivos gráficos (histograma y ajuste.)

\incluircsv{final.csv}
           {|c|c|c|c|}
           {Muestra & $\sigma$ & $\beta$ & $\lambda$}
           {\nombremuestra{\Muestra} & \Sigma & \Beta & \Lambda}

\graficodoble{X1-final.tex}{Muestra X1}{X2-final.tex}{Muestra X2}
\graficodoble{X3-final.tex}{Muestra X3}{X4-final.tex}{Muestra X4}
\graficodoble{X5-final.tex}{Muestra X5}{X6-final.tex}{Muestra X6}
\grafico{X7-final.tex}{Muestra X7}


%%%%%%%%%%%%%%%%%%%%%%%%%%%%%%%%%%%%%%%%%%%%%%%%%%%%%%%%%%%%%%%%%%%%%%%%%%%%%%%
%% Discusión                                                                 %%
%%%%%%%%%%%%%%%%%%%%%%%%%%%%%%%%%%%%%%%%%%%%%%%%%%%%%%%%%%%%%%%%%%%%%%%%%%%%%%%


\section{Discusión}

\blindtext


%%%%%%%%%%%%%%%%%%%%%%%%%%%%%%%%%%%%%%%%%%%%%%%%%%%%%%%%%%%%%%%%%%%%%%%%%%%%%%%
%% Conclusiones                                                              %%
%%%%%%%%%%%%%%%%%%%%%%%%%%%%%%%%%%%%%%%%%%%%%%%%%%%%%%%%%%%%%%%%%%%%%%%%%%%%%%%


\section{Conclusiones}

\blindtext


%%%%%%%%%%%%%%%%%%%%%%%%%%%%%%%%%%%%%%%%%%%%%%%%%%%%%%%%%%%%%%%%%%%%%%%%%%%%%%%
%% Apéndice A: Enunciado del Trabajo Práctico                                %%
%%%%%%%%%%%%%%%%%%%%%%%%%%%%%%%%%%%%%%%%%%%%%%%%%%%%%%%%%%%%%%%%%%%%%%%%%%%%%%%


\section{Apéndice A: Enunciado del Trabajo Práctico}

\newcommand{\real}{\mathbb{R}}
\newcommand{\nat}{\mathbb{N}}
\newcommand{\eme}{\mathcal{M}}
\newcommand{\emeh}{\widehat{\mathcal{M}}}
\newcommand{\ere}{\mathcal{R}}

\subsection{Introducci\'on}

Numerosas aplicaciones utilizan la distribuci\'on Gamma Generalizada (D$\Gamma$G) para modelar datos, cuya funci\'on de densidad se expresa como $$f_\Theta(x)=\frac{\beta\, x^{\beta \lambda - 1}}{\sigma^{\beta \lambda}\, \Gamma(\lambda)} \exp\left\{-\left(\frac{x}{\sigma} \right)^\beta \right\}\qquad\textrm{con } x\in\real_{>0}$$ donde 
$\Gamma(\cdot)$ es la funci\'on Gamma\footnote{La funci\'on Gamma, en el caso entero, es equivalente a la funci\'on factorial: $\Gamma(n)=n!$, $\forall n\in\nat$ } definida como $\Gamma(z)=\int_0^{\infty}{t^{z-1}e^{-t}\,dt}$, y $\Theta$ representa a la tupla de par\'ametros $\Theta=(\sigma,\beta,\lambda)$. El primero se encuentra relacionado con la escala de la funci\'on $f$ y los otros dos con la forma; todos son positivos.
La D$\Gamma$G engloba un conjunto amplio de distribuciones param\'etricas, donde la distribuci\'on exponencial, Weibull o Gamma son casos especiales de \'esta.
En vez de utilizar (y almacenar) el histograma emp\'irico de los datos, representarlos solamente con los par\'ametros de esta distribuci\'on resulta ser, en muchos casos, una opci\'on m\'as que conveniente. El problema que surge consiste en estimar de forma certera y eficiente los par\'ametros de la D$\Gamma$G que mejor ajusta a los datos.

\subsection{Estimaci\'on de par\'ametros}

Sean $n$ datos reales positivos $x_1,\dots,x_n$. Estos datos pueden provenir de mediciones de un sat\'elite, p\'ixeles de im\'agenes, etc., y de los cuales asumimos que siguen una distribuci\'on D$\Gamma$G cuyos par\'ametros queremos estimar. Como se vi\'o en la materia \emph{Proba}, existen m\'etodos como `\textsl{estimadores de m\'axima verosimilitud'}, donde a partir de las muestras se pueden estimar par\'ametros de distribuciones. Utilizando este m\'etodo, llegamos a que se deben cumplir las siguientes condiciones:
\begin{eqnarray}
\tilde{\sigma} & = & \left(\frac{\sum_{i=1}^n{x_i^{\tilde{\beta}}}}{n \tilde{\lambda}} \right)^{1/\tilde{\beta}} \label{eqn:sigma}\\
\tilde{\lambda} & = &\left[ \tilde{\beta} \left( \frac{\sum_{i=1}^n{x_i^{\tilde{\beta}} \log x_i}}{\sum_{i=1}^n{x_i^{\tilde{\beta}}}} - \frac{\sum_{i=1}^n{\log x_i}}{n} \right)\right]^{-1} \label{eqn:lambda}\\
 0 & = & \frac{\tilde{\beta}}{n} \sum_{i=1}^n{\log x_i} - \log {\sum_{i=1}^n{x_i^{\tilde{\beta}}}} + \log(n \tilde{\lambda}) - \psi(\tilde{\lambda})
\end{eqnarray}
donde $\tilde{\Theta}=(\tilde{\sigma},\tilde{\beta},\tilde{\lambda})$ son las estimaciones a partir de las muestras de datos y $\psi(\cdot)$ es la `conocida' funci\'on digamma\footnote{$\psi(z) = \frac{\partial}{\partial z} \ln \Gamma (z) = \frac{\Gamma'(z)}{\Gamma(z)}$.} que se define como la derivada logar\'itmica de la funci\'on Gamma. Notar que una una vez estimado el par\'ametro $\beta$, los otros dos par\'ametros pueden obtenerse de las primeras dos ecuaciones. A partir del c\'alculo de los \textsl{momentos} de una variable aleatoria con funci\'on de densidad $f_\Theta(x)$ se pueden obtener las siguientes dos ecuaciones:
\begin{eqnarray}
\log(\eme(2\beta)) -2\log(\eme(\beta)) & = & \log(1+\beta(\ere(\beta)-\ere(0))) \label{eqn:sol1}\\
\frac{\eme(2\beta)}{\eme^2(\beta)} & = & 1+\beta(\ere(\beta)-\ere(0))) \label{eqn:sol2}
\end{eqnarray}
donde 
$$\eme(s) =  \frac{1}{n}\sum_{i=1}^n{x_i^s}\,, \qquad \emeh(s) = \frac{1}{n}\sum_{i=1}^n{x_i^s \log(x_i)}\,,  \qquad \ere(s) = \frac{\emeh(s)}{\eme(s)}$$
Estas ecuaciones tienen como ventaja que no dependen de los otros par\'ametros, utiliz\'andose funciones m\'as sencillas que s\'olo dependen de los datos. Hallando la soluci\'on de cualquiera de las ecuaciones (\ref{eqn:sol1}) o (\ref{eqn:sol2}), es posible estimar el par\'ametro $\beta$, y despejar el resto de los par\'ametros utilizando las ecuaciones (\ref{eqn:sigma}) y (\ref{eqn:lambda}).

\subsection{Enunciado}

El objetivo del trabajo pr\'actico es implementar un programa que permita estimar los par\'ametros $\Theta=(\sigma,\beta,\lambda)$ a partir de un conjunto de $n$ datos. Para ello, se deber\'a resolver la ecuaciones (\ref{eqn:sol1}) o (\ref{eqn:sol2}).
Evaluando los distintos m\'etodos vistos en clase que permitan resolver este problema, se deber\'a realizar una implementaci\'on cumpliendo lo siguiente:

\begin{enumerate} 
\item Implementar al menos dos m\'etodos (de los cuales uno de ellos debe ser el m\'etodo de Newton) con aritm\'etica binaria de punto flotante con $t$ d\'igitos de precisi\'on en la mantisa. El valor $t$ debe ser un par\'ametro de la implementaci\'on, con $t<52$.
\item  Realizar experimentos num\'ericos con cada m\'etodo implementado en el \'item anterior elegiendo varias instancias de prueba y en funci\'on de las cantidad de d\'igitos $t$ de precisi\'on en la mantisa (experimentar con al menos 3 valores distintos de $t$).
\item Para cada m\'etodo implementado se deber\'an mostrar resultados obtenidos en cuanto a cantidad necesaria de iteraciones, tiempo de ejecuci\'on, precisi\'on en el resultado, y cualquier otro par\'ametro que considere de inter\'es evaluar.
\item Realizar el gr\'afico del histograma de los datos y el ajuste obtenido. Extraer conclusiones sobre la efectividad de cada m\'etodo observando los resultados anteriores.  
\end{enumerate}

\subsection{Formato de archivos de entrada}

El programa debe tomar los datos desde un archivo de texto con el siguiente formato:

\vspace{0.5cm}

\begin{tabular}{|l|} \hline 
\verb|input_file.txt|\\ \hline
\verb|n|\\ 
$\mathtt{x_1\ x_2\ \ldots\ x_n}$ \\ \hline
\end{tabular}

\vspace{0.5cm}

El archivo contiene en la primer l\'inea la cantidad de datos, y en la l\'inea siguiente se encuentran los $n$ datos (reales positivos) separados por espacio. En la web de la materia se publicar\'an varios archivos de prueba para realizar los primeros experimentos.

\subsection{Fecha de entrega:} 

\begin{itemize}
\item \textsl{Formato electr\'onico:} domingo 14 de abril de 2013, hasta las 23:59 hs., enviando el trabajo (informe+c\'odigo) a \texttt{metnum.lab@gmail.com}. El subject del email debe comenzar con el texto \verb|[TP1]| seguido de la lista de apellidos de los integrantes del grupo. 
\item \textsl{Formato f\'isico:} lunes 15 de abril de 2013, de 18 a 20hs (en la clase pr\'actica).
\end{itemize}


%%%%%%%%%%%%%%%%%%%%%%%%%%%%%%%%%%%%%%%%%%%%%%%%%%%%%%%%%%%%%%%%%%%%%%%%%%%%%%%
%% Apéndice B: Código Fuente                                                 %%
%%%%%%%%%%%%%%%%%%%%%%%%%%%%%%%%%%%%%%%%%%%%%%%%%%%%%%%%%%%%%%%%%%%%%%%%%%%%%%%


\section{Apéndice B: Código Fuente}
\subsection{Main.cpp}
\lstinputlisting[language=C++]{../src/Main.cpp}
\subsection{Metodos.h}
\lstinputlisting[language=C++]{../src/Metodos.h}
\subsection{Metodos.cpp}
\lstinputlisting[language=C++]{../src/Metodos.cpp}
\subsection{Ecuaciones.h}
\lstinputlisting[language=C++]{../src/Ecuaciones.h}
\subsection{Ecuaciones.cpp}
\lstinputlisting[language=C++]{../src/Ecuaciones.cpp}
\subsection{TFloat.h}
\lstinputlisting[language=C++]{../src/TFloat.h}
\subsection{TFloat.cpp}
\lstinputlisting[language=C++]{../src/TFloat.cpp}
\subsection{lib/getopt\_pp.h}
\lstinputlisting[language=C++]{../src/lib/getopt_pp.h}
\subsection{lib/getopt\_pp.cpp}
%\lstinputlisting[language=C++]{../src/lib/getopt_pp.cpp}
\end{document}